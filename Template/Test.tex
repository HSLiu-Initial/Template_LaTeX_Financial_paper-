\documentclass[UTF8]{ctexart}
%以下设置第一部分:字体和字号
\usepackage{fontspec}
\usepackage{xeCJK}
\usepackage{ctex}
    %设置整体的字体
    \setmainfont{Times New Roman}
    \setCJKmainfont{SimSun}

    %设置行间距
    \linespread{1.5}

    %设置论文标题格式
    %事实上LaTeX并没有修改格式的选项,如果需要修改,把字体命令直接放到\title \author等选项中即可
    %\title{\zihao{3}\textbf{你的标题}}

    %设置正文,摘要字号
    \zihao{-4}

    %设置1级标题,2级标题的字体字号
    \CTEXsetup[ name={,、},
                number={\chinese{section}},
                format={\raggedright\bfseries\zihao{4}},
                aftername={~}                          
                                                ]{section}
                                                
    \CTEXsetup[ name={(,)},
                number={\chinese{subsection}},
                format={\raggedright\bfseries\zihao{-4}}
                aftername={~}                              
                                                ]{subsection}

\title{BRILink项目提高印尼金融服务可获得性}
\author{温君南 \\ 曾燕}

\begin{document}
    \maketitle

    \section{案例简介}
    印尼人民银行 (Bank Rakyat Indonesia, BRI) 推出的BRILink项目是印尼“无分支代理 (Laku Pandai) ”项目 的多种形式之一,该项目在印尼全国范围内广泛招募代理商,通过他们向对应网点的客户提供基础的金融服务。BRILink项目为印尼人民银行拓展了金融服务的空间范围与客户群体,提高了印尼民众对金融服务的可获得性 (Nuryan et al., 2020)。本节对BRILink项目的运营模式、基本业务、安全保障措施与发展状况进行介绍。

    \subsection{BRILink项目的运营模式}
    BRILink项目以代理商网点为枢纽,构建了“银行——BRILink代理商网点——客户”的运营模式。如图1所示,BRILink代理商接受印尼人民银行的委托与监督,为在网点处办理业务的客户提供金融服务,并且帮助不熟悉业务流程的客户办理业务。代理商从代理点处每笔交易的手续费中抽取50\%的佣金。BRILink项目代理商网点广泛分布在社区与自然村内的便利店、邮局中,具有较高的渗透率。
\end{document}